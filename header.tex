\documentclass[
  paper=a4,
  headsepline,
  parskip=half,
  headheight=45pt,
]{scrartcl}

\usepackage[aux]{rerunfilecheck}

% deutsche Spracheinstellungen
\usepackage{polyglossia}
\setmainlanguage{german}

\usepackage{amsmath}
\usepackage{amssymb}
\usepackage{mathtools}

% Fonteinstellungen
\usepackage{fontspec}

\usepackage[
  math-style=ISO,    % \
  bold-style=ISO,    % |
  sans-style=italic, % | ISO-Standard folgen
  nabla=upright,     % |
  partial=upright,   % /
]{unicode-math}
\setmathfont{Latin Modern Math}

% richtige Anführungszeichen
\usepackage[autostyle]{csquotes}

% Zahlen und Einheiten
\usepackage[
  locale=DE,                   % deutsche Einstellungen
  separate-uncertainty=true,   % Immer Fehler mit \pm
  per-mode=symbol-or-fraction, % m/s im Text, sonst Brüche
]{siunitx}

% schöne Brüche im Text
\usepackage{xfrac}

\usepackage[
  labelfont=bf,        % Tabelle x: Abbildung y: ist jetzt fett
  font=small,          % Schrift etwas kleiner als Dokument
  width=0.9\textwidth, % maximale Breite einer Caption schmaler
]{caption}
\usepackage{subcaption}

\usepackage{graphicx}
% größere Variation von Dateinamen möglich
\usepackage{grffile}

% Standardplatzierung für Floats einstellen
\usepackage{float}
\usepackage{scrhack}
\floatplacement{figure}{htbp}
\floatplacement{table}{htbp}

% schöne Tabellen
\usepackage{booktabs}
\usepackage{multirow}

\usepackage{xparse}
\usepackage{xstring}

%Quellcode
\usepackage{exsheets-listings}
\usepackage{listings}
\usepackage{lstautogobble}
\lstset{
  autogobble=true,
  numbers=left,
  numberstyle=\tiny,
  numbersep=10pt,
  basicstyle=\ttfamily,
}

% Kopfzeile:
\usepackage{scrlayer-scrpage}


% Pakate für die Aufgaben
\usepackage{exsheets}
\DeclareTranslation{german}{exsheets-exercise-name}{Aufgabe}
\SetupExSheets{
  headings=block-subtitle,
  headings-format={\large\bfseries},
  subtitle-format={\large\itshape},
  points/format={\bfseries},
  counter-format = qu\IfQuestionSubtitleT{:},
}

\usepackage{enumitem}
\setlist[enumerate,1]{label=\bfseries\alph*)}

% Hyperlinks im Dokument
\usepackage[
  colorlinks,
  urlcolor=gray,
  unicode,
  pdfcreator={},  % PDF-Attribute säubern
  pdfproducer={}, % "
]{hyperref}
\usepackage{bookmark}

% Trennung von Wörtern mit Strichen
\usepackage[shortcuts]{extdash}


%% activate the solutions if jobname is solution
\IfStrEq{\jobname}{\detokenize{solution}}{% True branch
  \SetupExSheets[solution]{
    print=true,
  }
}{}

\setkomafont{pagehead}{\bfseries\upshape\large}
\makeatletter
\ihead{%
  \sheetnumber. Übungsblatt\\
  \@title
  \handindate
}
\chead{}
\ohead{%
  \@date \\ 
  \@author
}
\makeatother
